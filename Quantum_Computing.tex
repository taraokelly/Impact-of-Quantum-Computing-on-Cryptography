\section{Quantum Computing}
blah blah blah

\subsection{Shor's Algorithm}
Shor's Algorithm has a severe impact on cryptography, potentially breaking the public key cryptography systems used today \cite{Q_Alg}.
In 1994, Peter Shor established a quantum algorithm to find the prime factors of an integer in polynomial time (\(O(N^c)\) time) as opposed to the exponential time (\(O(c^N)\) time) of a classical computer with the same objective. 
The proficient factorization can be obtained as all of the values of a certain function can be computed by applying quantum parallelism and entanglement \cite{Impact_QC_Cryptog_1}. 
%Wright - "quantum alg has been dev to find the prime factors of an integer in polynomial time rather than exponential time. It uses quantum parallelism and entanglement to simultaneously compute all of the values of a certain function."
Shor has also released an adapted version that can solve the discrete logarithm problem \cite{Post_Q_Cryptog}, damaging the impregnability of the discrete logarithm problem and ECDLP. 

Public key encryption and digital signatures such as the favored algorithms RSA, ECC, Digital Signature Algorithm and Elliptic Curve Digital Signature Algorithm are vulnerable in this scenario. With the corruption of public key encryption comes the simultaneous weakening of single key encryption by straining the means to disclose symmetric keys.

\subsection{Grover's Algorithm}
Grover's Algorithm was formulated by Lov Grover in 1996. This algorithm is a strong foundation on which lot of the superior applications of quantum computing can be built \cite{Post_Q_Cryptog}. Harmful affects by the algorithm in question can be seen in symmetric encryption and hash functions, including the ever popular AES and the SHA family. It important to note that Grover's algorithm only decreases the security of these cryptographic methods; it does not break them.

A classical computer can do a search with the speed of \(O(N)\) assuming we do not know if the search parameter exists in an unordered domain. Grover's algorithm suggests that the same search can be implemented in \(O(\sqrt{N}\)) time using quantum queries. Grover's acceleration from \(O(N)\) to \(O(\sqrt{N}\)) is not quite as calamitous as Shor's. However, indicates another significant quantum approach to cryptanalysis \cite{Quantum_Cryptanal}, the study of code breaking. 

\subsection{Security Implications}
The exertion of the quantum algorithms above on a capable quantum computer has the power to revolutionize cryptanalysis. As Bacon and Dam stated, ``We can communicate securely, today, given that we cannot build a large scale quantum computer tomorrow" \cite{Q_Alg}. This extensive advancement in cryptanalysis will presumably open the door to a copious amount of privacy issues. Under the assumption of a relevant quantum computer, an unauthorized third party could eavesdrop on transactions and even manipulate said transactions. Additionally, this third party could easily impersonate others - even trusted sites, with the debilitation of digital certificates. The cryptographic principles confidence, integrity, authentication and non-repudiation would all be put in jeopardy. Many types of digital transactions would be called into question; personal banking, online purchases and private messaging to name a few. The immense volume of private information accumulated in our cloud based world, such as the vast amount of personal information submitted to social networks and medical records, would no longer be secure. 

The variety of reasons to motivate malicious hackers is extensive. For example, a company may be seeking to gain an unfair advantage over competitors or a government may wish to monitor transactions to identify and combat threats. A government's acquisition of a quantum cryptanalysis system can be seen as a compelling concern \cite{Sec_Risk}. The exploitation of citizen privacy may occur. The NSA's leaked documents revealed PRISM, a strict surveillance program, that can be referenced in this case. PRISM achieved extreme monitoring with the cooperation of service providers. A program such as PRISM can lead to a harmful surveillance society, possibly resulting in citizen unrest and mistrust of their governing authority. In a drastic instance, global tensions could be escalated by increasing a governments ability to spy on another. Whether it be by discovering another administration's private information, or the detection of another administration observing their state or their civilian's private information. 
Of course it is not only powerful entities, like a sizable company or a governing body, with the means to develop or obtain a quantum computer that are applicable to be considered a threat. One does not have to own a quantum computer, but could rely on a service provided by another party that possesses a quantum computer. 