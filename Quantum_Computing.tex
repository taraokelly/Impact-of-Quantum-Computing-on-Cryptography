\section{Quantum Computing}
Quantum computers apply quantum mechanics to perform computations. Quantum mechanics is the study of the laws of nature on an atomic and subatomic level. The behaviour at this scaled-down level can be replicated at certain temperatures; temperatures just above absolute zero as reported by Conover \cite{Q_Comp_Real}.

Unlike the binary bits of a classical computer, qubits utilize subatomic particles to represent data. 
This allows qubits the additional possibility of being in more than one state simultaneously.
A phenomenon known as a superposition. Therefore, a qubit can be in the state of either 0, 1 or a combination of both. Once a qubit is read, it is similar to a classical bit in that it can only be one of two states, 0 or 1.

Given a superposition, all states of the qubit can be operated on at the same time. This is called quantum parallelism and induces faster computations. So on an \(n\)-qubit computer, \(2n\) values can be computed \cite{Impact_QC_Cryptog}. The speed of a quantum computer can also be credited to it's entangled state \cite{Impact_QC_Cryptog_1}. 
Qubits can be entangled, a sort of relationship where the state of one qubit depends on the state of another. For instance, this enables us to determine the state of a qubit in an entangled pair by observing the state of the other qubit. 
However, the exploitation of quantum behaviour does not expedite the performance of all tasks \cite{Q_Comp_Real}. Only certain quantum algorithms for specific tasks are capable of surpassing the efficiency of classical algorithms. To date, the algorithms for factoring and searching by Shor and Grover are the most promising \cite{Impact_QC_Cryptog_1}.

\subsection{Shor's Algorithm}
Shor's Algorithm has a severe impact on cryptography, potentially breaking the public key cryptography systems used today \cite{Q_Alg}.
In 1994, Peter Shor established a quantum algorithm to find the prime factors of an integer in polynomial time (\(O(N^c)\) time) as opposed to the exponential time (\(O(c^N)\) time) of a classical computer with the same objective. 
The proficient factorization can be obtained as all of the values of a certain function can be computed by applying quantum parallelism and entanglement \cite{Impact_QC_Cryptog_1}. 
%Wright - "quantum alg has been dev to find the prime factors of an integer in polynomial time rather than exponential time. It uses quantum parallelism and entanglement to simultaneously compute all of the values of a certain function."
Shor has also released an adapted version thus damaging the impregnability of the discrete logarithm problem and ECDLP \cite{Post_Q_Cryptog}. 

Public key encryption and digital signatures such as the favored algorithms RSA, ECC, Digital Signature Algorithm and Elliptic Curve Digital Signature Algorithm are vulnerable in this scenario. With the corruption of public key encryption comes the simultaneous weakening of single key encryption by straining the means to disclose symmetric keys.

\subsection{Grover's Algorithm}
Grover's Algorithm was formulated by Lov Grover in 1996. This algorithm is a strong foundation on which lot of the superior applications of quantum computing can be built \cite{Post_Q_Cryptog}. Harmful effects by the algorithm in question can be seen in symmetric encryption and hash functions, including the ever popular AES and the SHA family. It is important to note that Grover's algorithm only decreases the security of these cryptographic methods; it does not break them.

A classical computer can do a search with the speed of \(O(N)\) assuming we do not know if the search parameter exists in an unordered domain. Grover's algorithm suggests that the same search can be implemented in \(O(\sqrt{N}\)) time using quantum queries. Grover's acceleration from \(O(N)\) to \(O(\sqrt{N}\)) is not quite as calamitous as Shor's. However, it indicates another significant quantum approach to cryptanalysis \cite{Quantum_Cryptanal}, the study of code breaking. 

\subsection{Security Implications}
The exertion of the quantum algorithms above on a capable quantum computer has the power to revolutionize cryptanalysis. Fortunately, no quantum computer is qualified to do so yet. As Bacon and Dam stated, ``We can communicate securely, today, given that we cannot build a large scale quantum computer tomorrow" \cite{Q_Alg}. This extensive advancement in cryptanalysis will presumably open the door to a copious amount of privacy issues. 

Under the assumption of a relevant quantum computer, an unauthorized third party could eavesdrop on transactions and even manipulate said transactions. Additionally, this third party could easily impersonate others - even trusted sites, with the debilitation of digital certificates \cite{Sec_Risk}. The cryptographic principles confidence, integrity, authentication and non-repudiation would all be put in jeopardy. Many types of digital transactions would be called into question; personal banking, online purchases and software downloads/updates to name a few. Not to mention the immense volume of private information accumulated in our cloud based world, such as the vast amount of personal information submitted to social networks and held in medical records, would no longer be secure. 

The variety of reasons to motivate malicious hackers is extensive. For example, a company may be seeking to gain an unfair advantage over competitors or a government may wish to monitor transactions to identify and combat threats. The numerous different possibilities of hackers, motivations and victims each spark worries over any conceivable collateral ramifications. A government's acquisition of a quantum cryptanalysis system can be seen as a particularly compelling concern. The exploitation of citizen privacy may occur. The NSA's leaked documents \cite{Sec_Risk} revealed PRISM, a strict surveillance program, that can be referenced in this case. PRISM achieved extreme monitoring with the cooperation of service providers. A program such as PRISM can lead to a harmful surveillance society, possibly resulting in citizen unrest and mistrust of their governing authority. In a drastic instance, global tensions could be escalated by increasing a governments ability to spy on another. Whether it be by discovering another administration's private information, or the detection of another administration observing their state or their civilian's private information. The very same leaked documents also exposed a plan to construct a cryptanalytically useful quantum computer.
Of course it is not only powerful entities, like a sizable company or a governing body, with the means to develop or obtain a quantum computer that are applicable to be considered a threat. One does not have to own a quantum computer, but could rely on a service provided by another party that possesses a quantum computer. 

\subsection{Post-Quantum Cryptography}
The security implications of quantum computers has seen to the investigation of quantum resistant cryptography. The less worrisome weakening of symmetric key cryptography can be addressed by increasing the security level. If not doing so already, Bernstein and Lange recommend simply switching to AES 256-bit keys \cite{Post_Q_Cryptog}. 

Tackling asymmetric cryptography is another matter.
Given that no relevant computer has been made yet, hope is not lost. 
Cryptographers worldwide have been attempting to find a solution to this imminent threat. 
Both classical and quantum solutions have been explored. Many solutions have been proposed, but only a handful of these have persevered through the extensive scrutiny.
It is difficult to actuate the strength of these proposals without the meticulous means to test them, e.g. a quantum computer at sufficient potential, time. 

%As synopsized by Bernstein and Lange \cite{Post_Q_Cryptog}, code-based encryption offers high confidence, fast encryption and short outputs. Unfortunately, the public keys are terribly large.
%The speed of lattice-based encryption/signatures are also favourable. Both short keys and outputs are produced. However, lattice-based cryptography requires more security analysis.
%Similarly, multivariate-quadratic-equation signatures grant short outputs, but also needs more security analysis.
As synopsized by Bernstein and Lange, code-based encryption, lattice-based encryption/signatures, multivariate-quadratic-equation signatures and hash-based signatures are practical classical proposals \cite{Post_Q_Cryptog}. However, none of these classical proposals are without shortcomings. For example, the likes of code-based encryption produces terribly large public keys \cite{Post_Q_Cryptog}, hash-based signatures are burdened with large signatures \cite{Sec_Risk} and as some of them are quite new, they are in need of additional time to further endure lengthy cryptanalysis \cite{Sec_Risk}.


%Hash-based signatures
Ironically, the same technology expected to suppress current security techniques also harbors great promise in bringing forth strong replacements \cite{Q_Cryptog}. 
%Most well studied is the Quantum Key Distribution Protocol (QKDP). 
Quantum Key Distribution (QKD) protocols make use of photons (particles of light) as opposed to computational complexity to distribute symmetric keys \cite{Impact_QC_Cryptog_1}. QKD is already commercially available, but only over limited distances, e.g. ID Quantique \cite{IDQ}. The first QKD protocol was published in 1984 by Bennett and Brassard, respectively earning the name BB84. Thereafter came Ekert's similarly named E91 and Bennett's solo B92 protocol \cite{Q_Cryptog}.
