\section{Security Implications of Quantum Computers}
blah

Does not have to have a QC but could be relying on a service provided by another party that has a quantum computer.

\subsection{Shor's Algorithm}
Shor's Algorithm has a severe impact on cryptography, potentially breaking the public key cryptography systems used today.
In 1994, Peter Shor established a quantum algorithm to find the prime factors of an integer in polynomial time (\(O(N^c)\) time) as opposed to the exponential time (\(O(c^N)\) time) of a classical computer with the same objective. 
The proficient factorization can be obtained as all of the values of a certain function can be computed by applying quantum parallelism and entanglement \cite{Impact_QC_Cryptog_1}. 
%Wright - "quantum alg has been dev to find the prime factors of an integer in polynomial time rather than exponential time. It uses quantum parallelism and entanglement to simultaneously compute all of the values of a certain function."
Shor has also released an adapted version that can solve the discrete logarithm problem \cite{Post_Q_Cryptog}, damaging the impregnability of the discrete logarithm problem and ECDLP. 

Public key encryption and digital signatures such as the favored algorithms RSA, ECC, Digital Signature Algorithm and Elliptic Curve Digital Signature Algorithm are vulnerable in this scenario. With the corruption of public key encryption comes the simultaneous weakening of single key encryption by straining the means to disclose symmetric keys.

\subsection{Grover's Algorithm}
Grover's Algorithm was formulated by Lov Grover in 1996. This algorithm is a strong foundation on which lot of the superior applications of quantum computing can be built \cite{Post_Q_Cryptog}. Harmful affects by the algorithm in question can be seen in symmetric encryption and hash functions, including the ever popular AES and the SHA family. It important to note that Grover's algorithm only decreases the security of these cryptographic methods; it does not break them.

A classical computer can do a search with the speed of \(O(N)\) assuming we do not know if the search parameter exists in an unordered domain. Grover's algorithm suggests that the same search can be implemented in \(O(\sqrt{N}\)) time using quantum queries. Grover's acceleration from \(O(N)\) to \(O(\sqrt{N}\)) is not quite as calamitous as Shor's. However, indicates another significant quantum approach to cryptanalysis \cite{Quantum_Cryptanal}.

\subsection{Social Concerns}
Hackers.

Governments:

Citizen Privacy - use NSA's leakage of PRISM program as example. Refer to \cite{Sec_Risk}.
    
Increase Global Tensions by increasing a governments ability to spy on another. 

    