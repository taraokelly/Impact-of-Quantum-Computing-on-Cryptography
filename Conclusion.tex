\section{Conclusion}
%It is quite apparent that quantum computers has already left it's mark on cryptography. And if not challenged, could turn that mark into a deep wound.
According to Kudelski there is no need to worry about quantum computers breaking current cryptographic systems for years to come; cryptographers have time to devise an optimal solution \cite{Impact_QC_Cryptog}. However, one must also consider the time costs of integration and standardization. It could take companies years to fully update their systems. Classical post-quantum solutions are complex, therefore costly to implement \cite{Sec_Risk}. Due to the uncertainty of quantum computers, entities are understandably hesitant to hastily invest in a new security system. Furthermore, for quantum cryptography to become commercially viable, the quantum internet would have to be more efficient and expansive. This depends on the advances in physics and technology \cite{Q_Cryptog}. Although, with a quantum satellite already orbiting the earth, the eventual future does look prosperous.

So can our security systems be quantum safe on time? Post-quantum cryptography is fighting to meet an unknown deadline, so there is no way to be sure. In fact, this review has been written under the assumption that the state of quantum computing is as publicly known. There is always the possibility that a capable quantum computer may be closer than we think. Not without regarding the extreme unlikeliness of the following, one may already exist. 